
\documentclass[preprint,12pt]{elsarticle}

\usepackage[spanish]{babel}
\usepackage{amssymb}
\usepackage{graphicx}
\usepackage{lineno}
\usepackage[utf8]{inputenc}
\usepackage{url}
\usepackage{natbib} 
\usepackage{amsmath} 
\usepackage{amssymb} 

\begin{document}
	
	\begin{frontmatter} 

		\title{\huge MESA DE AYUDA}
		
		\author{Huichi Contreras, Franklin Carlos            (2016056193)}
		\author{Gonzales Cave, Angel Gabriel              	(2017057861)}
		\author{Condori Quispe, Yhónn Joel	         	(2016056358)} 
		\author{Pastor Mendoza, José Edilberto             	(2016055237)} 
		\address{Escuela Profesional de Ingeniería de Sistemas}
		\address{Universidad Privada de Tacna}
		\address{Tacna, Perú}
		
%% ABSTRACT --------------------------------------------------------------------------------------------------------------------

		\begin{abstract}
The proposal the design a help desk for the technical support area of the Tacna Private University arises due to the demand of user requests to solve basic problems in their computer equipment, the general objective of this project is to provide the necessary information for the user and in turn provide the basic knowledge to solve technical problems in the equipment.

		\end{abstract}

%% ----------------------------------------------------------------------------------------------------------------------------------

	\end{frontmatter}

%% RESUMEN ---------------------------------------------------------------------------------------------------------------------

\section{Resumen}
La propuesta de diseñar una mesa de ayuda para el área de soporte técnico de la Universidad Privada de Tacna surge debido a la demanda de solicitudes de los usuarios para resolver problemas básicos en sus equipos informáticos, el objetivo general de este proyecto es proporcionar la información necesaria para El usuario y a su vez proporcionan los conocimientos básicos para la resolución de problemas técnicos en el equipo.


%% ----------------------------------------------------------------------------------------------------------------------------------


%% INTRODUCION ----------------------------------------------------------------------------------------------------------------

\section{Introducción} 

EDITAR\\

%% ----------------------------------------------------------------------------------------------------------------------------------


%% TITULO  ------------------------------------------------------------------------------------------------------------

\section{Titulo}

EDITAR\\

%% ----------------------------------------------------------------------------------------------------------------------------------

%% AUTORES  ------------------------------------------------------------------------------------------------------------

\section{Autores}

EDITAR\\

%% ----------------------------------------------------------------------------------------------------------------------------------

%% PLANTEAMIENTO DEL PROBLEMA ------------------------------------------------------------------------------------------------------------

\section{Planteamiento del problema}
Actualmente los gestores de mesa de ayuda o el personal de soporte técnico de la Universidad Privada de Tacna tienen que proporcionar la misma información a los diferentes usuarios, puesto que existe redundancia en problemas de soporte técnico básicos a nivel de hardware y software que se presentan y muchas veces es difícil atender todas las peticiones al mismo instante; lo cual demanda pérdida de tiempo en la solución de dichos problemas. A su vez las empresas  cuentan con una sola persona encargada de dar el soporte técnico que solicita el personal, dado que al existir demanda de problemas técnicos, la persona encargada tenga que atender los problemas presentados en el orden que lo asistieron, dejando al resto del personal en espera. El tiempo que demora el encargado de dar el soporte técnico a cada problema del personal, impide que el mismo pueda realizar el resto de sus actividades diarias, pues muchas veces el encargado de dar el soporte técnico al personal también es encargado de otras actividades de la empresa. 

%%  SUBSECCION 

\subsection {\textbf{Problema}}
Demora en solución de problemas de soporte tecnico a nivel de software y hardware. Suele suceder que los componentes de cómputo funcionen de forma incorrecta que causa un desempeño poco fiable del sistema, futuros fallos en el equipo y poca garantia en el servicio de la empresa.

EDITAR\\

%%  SUBSECCION 

\subsection {\textbf{Justificacion}}

EDITAR\\

%%Ejemplo de cita
\cite{Gartner} 

\begin{itemize}
	\item x
	\item y
	\item z
\end{itemize}

%%  SUBSECCION 

\subsection {\textbf{Alcance}}
El propósito de este proyecto es desarrollar un sistema de mesa de ayuda; este servicio permitira que el usuario obtenga la informacion que necesita para resolver algún inconveniente que se le presente, sin la necesidad de capacitarse en soporte técnico básico.

El sistema de mesa de ayuda permitirá que el tiempo en resolución de problemas técnicos sean resueltos a la brevedad posible por el usuario puesto que contara con un entorno que dispondra con las preguntas o problemas mas frecuentes, en caso sea necesario se solicitará la informacion del técnico de la empresa.

%% ----------------------------------------------------------------------------------------------------------------------------------

%% OBJETIVOS ------------------------------------------------------------------------------------------------------------

\section{Objetivos}

EDITAR\\

%% Ejemplo de inclusión de imagen
\begin{figure}[htb]
	\begin{center}
		% ESTA PARTE ME DA UN ERROR !!!
		%\includegraphics[width=14cm]{./IMAGENES/basededatos_1} 
		%\caption{Incluyendo la base de datos en DevOps}
	\end{center}
\end{figure}

%%  SUBSECCION 

\subsection{\textbf{General}}

EDITAR\\

\begin{itemize}

\item x
\item y
\item z

\end{itemize}

%%  SUBSECCION 

\subsection{\textbf{Especificos}}

EDITAR\\

%% ----------------------------------------------------------------------------------------------------------------------------------
 

%% REFERENTES TEORICOS ---------------------------------------------------------------------------------------------------

\section{Referentes Teoricos}

EDITAR\\

%% ----------------------------------------------------------------------------------------------------------------------------------


%% DESARROLLO DE LA PROPUESTA ---------------------------------------------------------------------------------------------------

\section{Desarrollo de la propuesta}

EDITAR\\

%%  SUBSECCION 

\subsection{\textbf{Tecnología de información}}

EDITAR\\

%%  SUBSECCION 

\subsection{\textbf{Metodología, técnicas usadas}}

EDITAR\\

%% ----------------------------------------------------------------------------------------------------------------------------------

%% CRONOGRAMA (PERSONAS, TIEMPO, OTROS RECURSOS) ---------------------------------------------------------------------------------------------------

\section{Cronograma}

EDITAR\\

%% ----------------------------------------------------------------------------------------------------------------------------------



%%  REFERENCIAS BIBLIOGRÁFICAS  (OJO NO TOCAR CON CITEP SOLO SE PONEN)------------------------------------------------------------------------------------------
	
	\newpage
	
	\bibliographystyle{apalike} 	%ESTILO
	\bibliography{BIBLIOGRAFIA}	 
	
	
\end{document}
