
\documentclass[preprint,12pt]{elsarticle}

\usepackage[spanish]{babel}
\usepackage{amssymb}
\usepackage{graphicx}
\usepackage{lineno}
\usepackage[utf8]{inputenc}
\usepackage{url}
\usepackage{natbib} 
\usepackage{amsmath} 
\usepackage{amssymb} 

\begin{document}
	
	\begin{frontmatter} 

		\title{\huge MESA DE AYUDA}
		
		\author{Huichi Contreras, Franklin Carlos            (2016056193)}
		\author{Gonzales Cave, Angel Gabriel                 (2017057861)}
		\author{Condori Quispe, Yhónn Joel	         	   (2016056358)} 
		\author{Pastor Mendoza, José Edilberto              (2016055237)} 
		\address{Escuela Profesional de Ingeniería de Sistemas}
		\address{Universidad Privada de Tacna}
		\address{Tacna, Perú}
		
%% ABSTRACT --------------------------------------------------------------------------------------------------------------------

		\begin{abstract}
The proposal the design a help desk for the technical support area of the Tacna Private University arises due to the demand of user requests to solve basic problems in their computer equipment, the general objective of this project is to provide the necessary information for the user and in turn provide the basic knowledge to solve technical problems in the equipment.

		\end{abstract}

%% ----------------------------------------------------------------------------------------------------------------------------------

	\end{frontmatter}

%% RESUMEN ---------------------------------------------------------------------------------------------------------------------

\section{Resumen}
La propuesta de diseñar una mesa de ayuda para el área de soporte técnico de la Universidad Privada de Tacna surge debido a la demanda de solicitudes de los usuarios para resolver problemas básicos en sus equipos informáticos, el objetivo general de este proyecto es proporcionar la información necesaria para El usuario y a su vez proporcionan los conocimientos básicos para la resolución de problemas técnicos en el equipo.


%% ----------------------------------------------------------------------------------------------------------------------------------


%% INTRODUCION ----------------------------------------------------------------------------------------------------------------

\section{Introducción} 

El presente trabajo, estar enfocado a mejorar los tiempos de respuesta y la asistencia oportuna, permitiendo el desarrollo normal de actividades propias de los procesos de cada área.

Sabemos que hoy por hoy,  los equipos informáticos son herramientas indispensables en cualquier organización , ya que nos facilita la realización de distintas actividades laborales y que es necesario que estos equipos funcionen adecuamente durante todo el tiempo que se requiera en cualquier momento.

Para dar respuestas a incidentes ligados a los equipos informáticos o activos tanto de hardware y software cada activo ha sido en algún momento instalado , configurado o diseñado por una persona el cual será sindicado como el responsable para dar solución al problema en cuestión.

Es vital realizar un reconocimiento de soluciones al historial de problemas que se puedan sucitar en las diferentes áreas  para así encontrar la solución óptima y oportuna tanto en la infraestructura ya sean los equipos de cómputo y también en la asistencia a posibles fallas de estos elementos.

Este proyecto de implementación de una mesa de ayuda en el área de soporte técnico de la Universidad Privada de Tacna pretende garantizar la eficiencia y eficacia en el soporte y asistencia de consultas con la finalidad de  tener un  normal desarrollo de las actividades propias de la organización.

%% ----------------------------------------------------------------------------------------------------------------------------------


%% TITULO  ------------------------------------------------------------------------------------------------------------

\section{Título}

DISEÑO E IMPLEMENTACIÓN DE MESA DE AYUDA PARA EL ÁREA DE SOPORTE TÉCNICO DE LA UNIVERSIDAD PRIVADA DE TACNA

%% ----------------------------------------------------------------------------------------------------------------------------------

%% AUTORES  ------------------------------------------------------------------------------------------------------------

\section{Autores}
Huichi Contreras, Franklin Carlos 

Gonzales Cave, Angel Gabriel

Condori Quispe, Yhónn Joel	
 
Pastor Mendoza, José Edilberto


%% ----------------------------------------------------------------------------------------------------------------------------------

%% PLANTEAMIENTO DEL PROBLEMA ------------------------------------------------------------------------------------------------------------

\section{Planteamiento del problema}
Actualmente los gestores de mesa de ayuda o el personal de soporte técnico de la Universidad Privada de Tacna tienen que proporcionar la misma información a los diferentes usuarios, puesto que existe redundancia en problemas de soporte técnico básicos a nivel de hardware y software que se presentan y muchas veces es difícil atender todas las peticiones al mismo instante; lo cual demanda pérdida de tiempo en la solución de dichos problemas. A su vez las empresas  cuentan con una sola persona encargada de dar el soporte técnico que solicita el personal, dado que al existir demanda de problemas técnicos, la persona encargada tenga que atender los problemas presentados en el orden que lo asistieron, dejando al resto del personal en espera. El tiempo que demora el encargado de dar el soporte técnico a cada problema del personal, impide que el mismo pueda realizar el resto de sus actividades diarias, pues muchas veces el encargado de dar el soporte técnico al personal también es encargado de otras actividades de la empresa. 

%%  SUBSECCION 

\subsection {\textbf{Problema}}
Demora en solución de problemas de soporte tecnico a nivel de software y hardware. Suele suceder que los componentes de cómputo funcionen de forma incorrecta que causa un desempeño poco fiable del sistema, futuros fallos en el equipo y poca garantia en el servicio de la empresa.

%%  SUBSECCION 

\subsection {\textbf{Justificacion}}

Las áreas de la Universidad Privada de Tacna en general vienen realizando diversas y diferentes tareas que estan ligadas a los procesos academicos y administrativos y también a la mejora continua de su imagen institucional.
En cumplimiento a sus funciones la Oficina de Soporte Técnico debe garantizar en todo momento de manera óptima la prestación de un servicio de calidad para continuar con las actividades en caso de la generación de para que el personal pueda ejecutar sus funciones.
 
Es necesario concentrar por medio de una mesa de ayuda, la atención de las solicitudes relacionadas con el hardware  software teniendo un inventario de equipos de cómputo actualizado, brindando soluciones a las solicitudes generadas por los usuarios, facilitando así una atención rápida y oportuna, garantizando la continuidad de las actividades del personal.

Con este servicio los usuarios podrán aclarar sus dudas y solucionar problemas ténicos basicos.

Ya no pasarán mucho tiempo esperando para que un ténico solucione los problemas, los clientes tendran un servicio a la vanguardia de la tecnologia y de esta manera la perdida de tiempo, las consultas repetitivas podran solventarse en menos tiempo.

Este servicio permitirá a su vez ir capacitando el conocimiento del usuario en soporte técnico de hardware y software; cuando vuelva a tener el mismo inconveniente con su equipo segurirá las pautas que le proporciono el sistema, evitando llamar al encargado de dar soporte a los usuarios y de esta manera el tiempo que se pierde tanto del encargado de dar soporte como del usuario quedará reducido.

El sistema no solo beneficiara al gestor de mesa de ayuda encargado del soporte tecnico sino tambien al usuario y al avance de las empresas, puesto que mientras menos tiempo se demande en resolver los problemas a nivel básico de soporte técnico en los equipos, más tiempo tiene el usuario de seguir trabajando en la actividad de la empresa.

%%Ejemplo de cita
\cite{Gartner} 

\begin{itemize}
	\item x
	\item y
	\item z
\end{itemize}

%%  SUBSECCION 

\subsection {\textbf{Alcance}}
El propósito de este proyecto es desarrollar un sistema de mesa de ayuda; este servicio permitira que el usuario obtenga la informacion que necesita para resolver algún inconveniente que se le presente, sin la necesidad de capacitarse en soporte técnico básico.

El sistema de mesa de ayuda permitirá que el tiempo en resolución de problemas técnicos sean resueltos a la brevedad posible por el usuario puesto que contara con un entorno que dispondra con las preguntas o problemas mas frecuentes, en caso sea necesario se solicitará la informacion del técnico de la empresa.

%% ----------------------------------------------------------------------------------------------------------------------------------

%% OBJETIVOS ------------------------------------------------------------------------------------------------------------

\section{Objetivos}
%% Ejemplo de inclusión de imagen
\begin{figure}[htb]
	\begin{center}
		% ESTA PARTE ME DA UN ERROR !!!
		%\includegraphics[width=14cm]{./IMAGENES/basededatos_1} 
		%\caption{Incluyendo la base de datos en DevOps}
	\end{center}
\end{figure}

%%  SUBSECCION 

\subsection{\textbf{General}}

Desarrollar un servicio de mesa de ayuda para la Universidad Privada de tacna con el fin de informar al usuario en los problemas mas frecuentes, mediante el uso de herramientas informáticas.

%%  SUBSECCION 

\subsection{\textbf{Especificos}}

\begin{itemize}

\item Analizar cuáles son las peticiones de soporte técnico mas solicitadas por los usuarios, para obtener la informacion necesaria y definir los requerimientos para el correcto desarrollo del sistema
\item Mejorar la productivdad de la empresa informando a los usuarios sobre como solucionar los problemas comunes.
\end{itemize}


%% ----------------------------------------------------------------------------------------------------------------------------------
 

%% REFERENTES TEORICOS ---------------------------------------------------------------------------------------------------

\section{Referentes Teoricos}
Biblioteca de Infraestructura de Tecnologías de la Información (ITIL) : Es un marco de buenas prácticas y conceptos para la gestión de servicios de tecnologías de la información, el desarrollo de tecnologías de la información y las operaciones relacionadas con diversas áreas de servicios TI.

A continuación se describe la conformación actual de ITIL v3 :
\begin{itemize}
\item Libro 1  Estrategia del Servicio 
Propone tratar la gestión de servicios no sólo como una capacidad sino como un activo estratégico
\item Libro 2 Diseño del Servicio
Cubre los principios y métodos necesarios para transformar los objetivos estratégicos en portafolios de servicios y activos.
\item Libro 3 Transición del Servicio
Cubre el proceso de transición para la implementación de nuevos servicios o su mejora
\item Libro 4 Operación del Servicio
Cubre las mejores prácticas para la gestión del día a día en la operación del servicio.
\item Libro 5 Mejora Continua del Servicio
Proporciona una guía para la creación y mantenimiento del valor ofrecido a los clientes a través de un diseño, transición y operación del servicio optimizado.
\end{itemize}
ITIL es un marco de referencia de las buenas prácticas, las que permiten incorporar estándares en los procesos, métodos y actividades existentes orientadas a un entorno de calidad en gestión de servicios TI, para satisafacer las necesidades y requerimientos de los clientes.

%% ----------------------------------------------------------------------------------------------------------------------------------


%% DESARROLLO DE LA PROPUESTA ---------------------------------------------------------------------------------------------------

\section{Desarrollo de la propuesta}

EDITAR\\

%%  SUBSECCION 

\subsection{\textbf{Tecnología de información}}

\begin{itemize}
\item Python es un lenguaje de programación interpretado cuya filosofía hace hincapié en una sintaxis que favorezca un código legible. Se trata de un lenguaje de programación multiparadigma, ya que soporta orientación a objetos, programación imperativa y, en menor medida, programación funcional. Es un lenguaje interpretado, dinámico y multiplataforma.

\item Flask es un framework minimalista escrito en Python que permite crear aplicaciones web rápidamente y con un mínimo número de líneas de código.

\item MySQL es un sistema de gestión de bases de datos, está considerada como la base de datos de código abierto más popular del mundo.
\end{itemize}

%%  SUBSECCION 

\subsection{\textbf{Metodología, técnicas usadas}}

SCRUM es una metodologÌa para gestion, mejora y mantenimiento de un sistema nuevo o existente. SCRUM se concentra en como los miembros del equipo deberÌan funcionar a fin de producir un sistema flexible en un entorno que cambia constantemente. 


%% ----------------------------------------------------------------------------------------------------------------------------------

%% CRONOGRAMA (PERSONAS, TIEMPO, OTROS RECURSOS) ---------------------------------------------------------------------------------------------------

\section{Cronograma}

\begin{figure}[htb]
	\begin{center}
		\includegraphics[width=14cm]{./IMAGENES/Gannt} 
		\caption{Cronograma del proyecto}
	\end{center}
\end{figure}

%% ----------------------------------------------------------------------------------------------------------------------------------



%%  REFERENCIAS BIBLIOGRÁFICAS  (OJO NO TOCAR CON CITEP SOLO SE PONEN)------------------------------------------------------------------------------------------
	
	\newpage
	
	\bibliographystyle{apalike} 	%ESTILO
	\bibliography{BIBLIOGRAFIA}	 
	
	
\end{document}
